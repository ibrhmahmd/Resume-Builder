%-------------------------------------------------------------------------------
%	SECTION TITLE
%-------------------------------------------------------------------------------
\cvsection{Projects}

%-------------------------------------------------------------------------------
%	CONTENT
%-------------------------------------------------------------------------------

\begin{cventries}
%---------------------------------------------------------

\cventry
    {OllamaNet is a modular, LLM platform enabling users to converse with, manage, and explore LLMs
    via independently deployable micro-services.
    Core objectives include model lifecycle management, Context exploration, and scalable, resilient architecture.}
    {OllamaNet — \href{https://github.com/ollamaNet/OllamaNet-Architecture}{\color{midnightblue}\textbf{GitHub}} \textrm{\faArrowRight}}
    {--}
    {May 2025}
    {
      \begin{cvitems}
        \item \textbf{Architecture:}{ Modular micro-services using Clean Architecture, Redis cache, RabbitMQ, and a central Gateway.
                                      Services include AuthService, ConversationService, ExploreService, and AdminService, 
                                      all communicating via HTTP and messaging.}
        \item \textbf{Patterns:}{ Repository, Resilience (Polly), Distributed Cache, Server-Sent Events for streaming.}
        \item \textbf{Features:}{ Context monetoring; model lifecycle management; fast model discovery; streaming via SSE; centralized documentation via Memory Bank; scalable and resilient core.}
        \item \textbf{Technical Skills:}{ .NET 9 (ASP.NET Core), Ocelot, SQL Server, EF Core, Redis, RabbitMQ, FluentValidation, JWT, Swagger, Semantic Kernal.}
      \end{cvitems}
    }
    %-----
    \begin{singlespace}
% Your text here ----
----
\end{singlespace}
    %--------------------------------------------------------------------


% \cventry
%     {Streamlit web application for YouTube video downloading with format options, quality selection, and progress tracking.} % 
%     {Streamlit Video-Downloader — \href{https://github.com/ibrhmahmd/Youtube-videoDownloader}{\color{midnightblue}\textbf{GitHub}} \textrm{\faArrowRight}} % Project name
%     {Techno Future Academy} % Organisation
%     {APR 2025} % Date(s)
%     {
%       \begin{cvitems} % Description(s) bullet points
%         \item \textbf{Architecture:}{ Three-layer design with UI, Download Core, and Processor components using Factory, Observer, and Strategy patterns.}
%         \item \textbf{Features \& Security:}{ Format selection, progress tracking, URL sanitization, efficient file operations with error handling.}
%         \item \textbf{Technical Skills:}{ Python, Streamlit, Pytube, FFmpeg, Async operations, Git.}
%       \end{cvitems}
%     }
%     %--------------------------------------------------------------------


\cventry
    {Lets users deploy and expose Ollama LLM models from cloud notebooks
    as public APIs via ngrok, with RabbitMQ for service discovery.
    Emphasizes simplicity, multi-model support, and seamless remote access.}
    {LLM-Inference — \href{https://github.com/ibrhmahmd/Inference-Engine}{\color{midnightblue}\textbf{GitHub}} \textrm{\faArrowRight}}
    {--}
    {May 2025}
    {
      \begin{cvitems}
        \item \textbf{Architecture:}{ Notebook-first orchestration of Ollama (local LLM server), ngrok (public tunneling), and RabbitMQ (service discovery).}
        \item \textbf{Features:}{ Deploy and serve multiple Ollama models; public API via ngrok; RabbitMQ-based service discovery; structured notebook cells for setup, deployment, and operation.}
        \item \textbf{Performance:}{ Functional prototype with successful multi-model support, public API exposure, and reliable RabbitMQ integration.}
        \item \textbf{Technical Skills:}{ Python, Jupyter Notebook, Ollama, ngrok, RabbitMQ, Uvicorn, pika, requests.}
      \end{cvitems}
    }
    %-----
    \begin{singlespace}
% Your text here ----
----
\end{singlespace}



\cventry
    {ConversationService is a core microservice in OllamaNet, 
    managing AI-powered conversations with real-time chat, message history, folder organization, note management, 
    feedback, and document processing (RAG).}
    {ConversationService — \href{https://github.com/ollamaNet/OllamaNet-Architecture/tree/master/ConversationService}{\color{midnightblue}\textbf{GitHub}} \textrm{\faArrowRight}}
    {--}
    {May 2025}
    {
      \begin{cvitems}
        \item \textbf{Architecture:}{ Modular, layered microservice. Implements domain-driven design, Clean Architecture, and robust caching.}
        \item \textbf{Features:}{ Real-time chat, persistent conversation history, folder and note management, document upload and RAG, caching, service Discovery.}
        \item \textbf{Performance:}{ Response time <100ms (cached), 99.9\% uptime; cache hit ratio >80\%.}
        \item \textbf{Technical Skills:}{ ASP.NET Core, EF Core, Redis, Pinecone, Semantic Kernel, RabbitMQ, Polly, FluentValidation.}
      \end{cvitems}
    }
    %-----
    \begin{singlespace}
% Your text here ----
----
\end{singlespace}


    %--------------------------------------------------------------------
\cventry
    {Speech-to-text solution using Google's Speech Recognition with FastAPI, supporting file uploads and microphone input.} % 
    {Speech-to-Text Transcription — \href{https://github.com/ibrhmahmd/audio-recognition}{\color{midnightblue}\textbf{GitHub}} \textrm{\faArrowRight}} % Project name
    {--} % Organisation
    {FEB 2025} % Date(s)
    {
      \begin{cvitems} % Description(s) bullet points
        \item \textbf{Architecture:}{ RESTful API using FastAPI with dedicated request handling and recognition logic layers.}
        \item \textbf{Features:}{ Audio transcription with WAV support, real-time processing, Arabic optimization, and async I/O operations.}
        \item \textbf{Technical Skills:}{ Python, FastAPI, REST APIs, Speech Recognition, Async Programming.}
      \end{cvitems}
    }
    %--------------------------------------------------------------------



\cventry
    {Web application for real estate companies to manage properties, tenants, and lease agreements with user-friendly portal.} %
    {Estate Agent — \href{https://github.com/ibrhmahmd/RealEstate}{\color{midnightblue}\textbf{GitHub}} \textrm{\faArrowRight}} % Project name
    {DEPI} % Organisation
    {SEPT 2024} % Date(s)
    {
      \begin{cvitems} % Description(s) bullet points
        \item \textbf{Architecture:}{ N-tier with ASP.NET Core MVC, SQL Server/EF Core using Repository pattern, and JWT authentication.}
        \item \textbf{Features:}{ Property management, tenant portal, lease agreements, online payments, and maintenance requests.}
        \item \textbf{Technical Skills:}{ C#, ASP.NET Core, EF Core, SQL Server, HTML/CSS/Bootstrap, jQuery.}
      \end{cvitems}
    }
    %--------------------------------------------------------------------


% \cventry
%     {CNN-based image classifier for flower categories using TensorFlow and Keras, achieving >90\% accuracy.} %
%     {TensorFlow Flower Classifier — \href{https://github.com/ibrhmahmd/Image-Recognition-using-TensorFlow}{\color{midnightblue}\textbf{GitHub}} \textrm{\faArrowRight}} % Project name
%     {University} % Organisation
%     {MAY 2024} % Date(s)
%     {
%       \begin{cvitems} % Description(s) bullet points
%         \item \textbf{Architecture:}{ Sequential CNN with three Conv+MaxPool blocks, dropout layers, and optimized for Adam with SparseCategoricalCrossentropy.}
%         \item \textbf{Data Processing:}{ Comprehensive image preprocessing with normalization and augmentation (flips, rotations, zooms).}
%         \item \textbf{Technical Skills:}{ Python, TensorFlow, Keras, NumPy, Matplotlib, CNN Design, Data Augmentation.}
%       \end{cvitems}
%     }
% %---------------------------------------------------------
% \end{cventries}


\begin{cventries}
%---------------------------------------------------------
% \cventry
%     {Classic Tetris game with C# and WPF featuring responsive controls and complete gameplay mechanics.} % 
%     {TETRIS — \href{https://github.com/ibrhmahmd/tetris}{\color{midnightblue}\textbf{GitHub}} \textrm{\faArrowRight}} % Project name
%     {DEPI} % Organisation
%     {July 2024} % Date(s)
%     {
%       \begin{cvitems} % Description(s) bullet points
%         \item \textbf{Architecture:}{ Object-oriented design with Model-View pattern, Factory Method for blocks, and GameState for state management.}
%         \item \textbf{Features:}{ Block movement/rotation, collision detection, line clearing, score tracking, and difficulty progression.}
%         \item \textbf{Technical Skills:}{ C#, WPF, XAML, .NET Framework, OOP, Game Development Principles.}
%       \end{cvitems}
%     }
% \end{cventries}
% %---------------------------------------------------------


%% The following section is commented out properly
%% \begin{singlespace}
%%   % Your text here ----
%% \end{singlespace}
%% \begin{cventries}
%% \cventry
%%     {Uses a genetic algorithm to evolve an image over generations, aiming to closely reproduce a target image. It leverages the pygad library for algorithm operations, demonstrating the optimization of image reproduction through evolutionary processes.} % 
%%     {Image Reproduction Using Genetic Algorithm — \href{https://github.com/ibrhmahmd/Genetic_Algorithm_Reproducing_Images}{\color{midnightblue}\textbf{GitHub}} \textrm{\faArrowRight}} % Project name
%%     {} % Organisation
%%     {July 2024} % Date(s)
%%     {
%%       \begin{cvitems} % Description(s) bullet points
%%         \item {Development:Developed customizable parameters for the genetic algorithm, allowing for easy adjustments.}
%%         \item {Achieved a 30 percent reduction in experiment setup time through the framework.}
%%         \item {Integrated a fitness function to assess image similarity.}
%%         \item {Developed a callback function to track progress during the algorithm's execution.}
%%         \item \textbf{Technical Skills:}{ Python, pygad, gari, numpy, imageio, matplotlib}
%%       \end{cvitems}
%%     }
%% \end{cventries}





