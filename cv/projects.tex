%-------------------------------------------------------------------------------
%	SECTION TITLE
%-------------------------------------------------------------------------------
\cvsection{Projects}

%-------------------------------------------------------------------------------
%	CONTENT
%-------------------------------------------------------------------------------

\begin{cventries}
%---------------------------------------------------------
\cventry
    {Advanced LLM Platform built with ASP.NET Core, providing a comprehensive microservices framework for Ollama LLM integration with enterprise-grade architecture, security, and scalability.
    } %
    {OllamaNet — \href{https://github.com/ollamaNet/OllamaNet_Components}{\color{midnightblue}\textbf{GitHub}} \textrm{\faArrowRight}} % Project name
    {Graduation Project} % Organisation
    {2025} % Date(s)
    {
      \begin{cvitems} % Description(s) bullet points
        \item \textbf{Architecture:}{ Microservices architecture with specialized components: API Gateway (Ocelot), Auth, Chat, Explore, Admin, and LLM Inference.}
        \item \textbf{Communication:}{ Implemented both synchronous (direct API calls) and asynchronous (RabbitMQ) communication patterns, JWT Redistoken caching, rate limiting}
        \item \textbf{Databases:}{ Shared approach with EF Core, Repository&UOW patterns, with service-specific schemas.}
        \item \textbf{Technical Skills:}{ ASP.NET Core, EF Core, Redis, JWT, RabbitMQ, Ocelot Gateway, Microservices, SQL Server, Git, Docker}
        \item \textbf{Soft Skills:}{ System Architecture Design, Technical Documentation, Team Management.}
      \end{cvitems}
    }
    %--------------------------------------------------------------------

\cventry
    {Streamlit web application for YouTube video downloading with format options, quality selection, and progress tracking.} % 
    {Streamlit Video-Downloader — \href{https://github.com/ibrhmahmd/Youtube-videoDownloader}{\color{midnightblue}\textbf{GitHub}} \textrm{\faArrowRight}} % Project name
    {Techno Future Academy} % Organisation
    {APR 2025} % Date(s)
    {
      \begin{cvitems} % Description(s) bullet points
        \item \textbf{Architecture:}{ Three-layer design with UI, Download Core, and Processor components using Factory, Observer, and Strategy patterns.}
        \item \textbf{Features \& Security:}{ Format selection, progress tracking, URL sanitization, efficient file operations with error handling.}
        \item \textbf{Technical Skills:}{ Python, Streamlit, Pytube, FFmpeg, Async operations, Git.}
      \end{cvitems}
    }
    %--------------------------------------------------------------------
      
\cventry
    {Local LLM server using Ollama framework for efficient deployment and interaction with open-source language models.} % 
    {LLM Deployment with Ollama — \href{https://github.com/ibrhmahmd/spicy-avocado}{\color{midnightblue}\textbf{GitHub}} \textrm{\faArrowRight}} % Project name
    {--} % Organisation
    {FEB 2025} % Date(s)
    {
      \begin{cvitems} % Description(s) bullet points
        \item \textbf{Model Deployment:}{ Deployed multiple LLMs (Qwen 2.5, Llama 3.2, Phi 3) with CUDA-based GPU acceleration for efficient inference.}
        \item \textbf{Server Infrastructure:}{ Integrated NGROK for secure tunneling, enabling remote access to the local LLM server.}
        \item \textbf{Technical Skills:}{ LLM Deployment, Python, REST APIs, CUDA, NGROK.}
      \end{cvitems}
    }
    %--------------------------------------------------------------------

\cventry
    {Speech-to-text solution using Google's Speech Recognition with FastAPI, supporting file uploads and microphone input.} % 
    {Speech-to-Text Transcription — \href{https://github.com/ibrhmahmd/audio-recognition}{\color{midnightblue}\textbf{GitHub}} \textrm{\faArrowRight}} % Project name
    {--} % Organisation
    {FEB 2025} % Date(s)
    {
      \begin{cvitems} % Description(s) bullet points
        \item \textbf{Architecture:}{ RESTful API using FastAPI with dedicated request handling and recognition logic layers.}
        \item \textbf{Features:}{ Audio transcription with WAV support, real-time processing, Arabic optimization, and async I/O operations.}
        \item \textbf{Technical Skills:}{ Python, FastAPI, REST APIs, Speech Recognition, Async Programming.}
      \end{cvitems}
    }
    %--------------------------------------------------------------------

\cventry
    {Web application for real estate companies to manage properties, tenants, and lease agreements with user-friendly portal.} %
    {Estate Agent — \href{https://github.com/ibrhmahmd/RealEstate}{\color{midnightblue}\textbf{GitHub}} \textrm{\faArrowRight}} % Project name
    {DEPI} % Organisation
    {SEPT 2024} % Date(s)
    {
      \begin{cvitems} % Description(s) bullet points
        \item \textbf{Architecture:}{ N-tier with ASP.NET Core MVC, SQL Server/EF Core using Repository pattern, and JWT authentication.}
        \item \textbf{Features:}{ Property management, tenant portal, lease agreements, online payments, and maintenance requests.}
        \item \textbf{Technical Skills:}{ C#, ASP.NET Core, EF Core, SQL Server, HTML/CSS/Bootstrap, jQuery.}
      \end{cvitems}
    }
    %--------------------------------------------------------------------

\cventry
    {CNN-based image classifier for flower categories using TensorFlow and Keras, achieving >90% accuracy.} %
    {TensorFlow Flower Classifier — \href{https://github.com/ibrhmahmd/Image-Recognition-using-TensorFlow}{\color{midnightblue}\textbf{GitHub}} \textrm{\faArrowRight}} % Project name
    {University} % Organisation
    {MAY 2024} % Date(s)
    {
      \begin{cvitems} % Description(s) bullet points
        \item \textbf{Architecture:}{ Sequential CNN with three Conv+MaxPool blocks, dropout layers, and optimized for Adam with SparseCategoricalCrossentropy.}
        \item \textbf{Data Processing:}{ Comprehensive image preprocessing with normalization and augmentation (flips, rotations, zooms).}
        \item \textbf{Technical Skills:}{ Python, TensorFlow, Keras, NumPy, Matplotlib, CNN Design, Data Augmentation.}
      \end{cvitems}
    }
%---------------------------------------------------------
\end{cventries}  

\begin{cventries}
%---------------------------------------------------------
\cventry
    {Classic Tetris game with C# and WPF featuring responsive controls and complete gameplay mechanics.} % 
    {TETRIS — \href{https://github.com/ibrhmahmd/tetris}{\color{midnightblue}\textbf{GitHub}} \textrm{\faArrowRight}} % Project name
    {DEPI} % Organisation
    {July 2024} % Date(s)
    {
      \begin{cvitems} % Description(s) bullet points
        \item \textbf{Architecture:}{ Object-oriented design with Model-View pattern, Factory Method for blocks, and GameState for state management.}
        \item \textbf{Features:}{ Block movement/rotation, collision detection, line clearing, score tracking, and difficulty progression.}
        \item \textbf{Technical Skills:}{ C#, WPF, XAML, .NET Framework, OOP, Game Development Principles.}
      \end{cvitems}
    }
%---------------------------------------------------------
\end{cventries}

\begin{singlespace}
  % Your text here ----
  \end{singlespace}
% \begin{cventries}
% %---------------------------------------------------------
%   \cventry
%     {Uses a genetic algorithm to evolve an image over generations, aiming to closely reproduce a target image. It leverages the pygad library for algorithm operations, demonstrating the optimization of image reproduction through evolutionary processes.
% } % 
%     {Image Reproduction Using Genetic Algorithm — \href{https://github.com/ibrhmahmd/Genetic_Algorithm_Reproducing_Images}{\color{midnightblue}\textbf{GitHub}} \textrm{\faArrowRight}}
% } % Project name
%     {} % Organisation
%     {July 2024} % Date(s)
%     {
%       \begin{cvitems} % Description(s) bullet points
%         \item {Development:Developed customizable parameters for the genetic algorithm, allowing for easy adjustments.}
%         \item {Achieved a 30 percent reduction in experiment setup time through the framework.}
%         \item {Integrated a fitness function to assess image similarity.
%                 }
%         \item {Developed a callback function to track progress during the algorithm's execution.}
%         \item \textbf{Technical Skills:}{ Python, pygad, gari, numpy, imageio, matplotlib}
%       \end{cvitems}
%     }
%---------------------------------------------------------
\end{cventries}